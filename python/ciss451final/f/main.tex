\newcommand\COURSENAME{Cryptography and Computer Security}
\newcommand\COURSESHORTNAME{Cryptography}
\newcommand\COURSENUMBER{CISS451/MATH451}
\newcommand\TITLE{Final Exam}
\newcommand\AUTHOR{Brandy Poag}

\documentclass[a4paper,12pt]{article}
\usepackage{assignment}
\definecolor{red}{rgb}{1,0,0}
\newcommand\redtext[1]{\textcolor{red}{#1}}
\begin{document}
\topmatter


The goal is to derive the formulas for the addition of points in 
a general elliptic curve:
\[
E: y^2 = x^3 + ax^2 + bx + c
\]
and implement an Python function to add points in an elliptic curve.

Write ${\cal O}$ for the point at infinity.

\newpage

Let $P = (x_1, y_1)$ and $Q = (x_2, y_2)$ be points on $E$.
(Therefore $P, Q$ are finite points.)
We want to derive the addition formulas for $R$ where $R = P + Q$.
Let $R = (x_3, y_3)$.

Q1. CASE: $x_1 \neq x_2$ (therefore $P \neq Q$)

(a) Let $L$ be the line through $P$ and $Q$ be
\[
L: y = \lambda x + \nu 
\] 
Derive $\lambda$ and $\nu$ in terms of $x_1, y_1, x_2, y_2$.

{\bf SOLUTION}


\begin{align*}
\lambda &= \frac{y_2 - y_1}{x_2 - x_1} \\
\\
\nu &= y_1 - \biggr( \frac{y_2 - y_1}{x_2 - x_1} \cdot x_1 \biggl)\\
\end{align*}



\newpage


(Continuing Q1.)

(b) Let $R' = (x'_3, y'_3)$ be the third point of intersection of $E$ and $L$.
Therefore $P, Q, R'$ must satisfy the equations of both $E$ and $L$:
\begin{align*}
y^2 &= x^3 + ax^2 + bx + c \\
y   &= \lambda x + \nu
\end{align*}
Through substitution, remove the variable $y$. 
You will obtain (of course) a cubic equation in $x$.
Derive and write down this equation in the form
\[
\text{cubic polynomial} = 0 
\]
This cubic polynomial of $x$ will have $a,b,c,\lambda, \nu$ 
in its coefficients.
(You need not substitute $\lambda$ and $\nu$ with their
expressions from (a).)

Of course the three solutions of $x$ gives the three $x$--coordinates
of $P$, $Q$, and $R'$.

\SOLUTION


\begin{align*}
0 &= x^3 + ax^2 + bx + c -{\lambda}^2x^2 - 2 \lambda \nu - {\nu}^2
\end{align*}


\newpage

(Continuing Q1.)

(c) We know that the equation (1) from (b) is satisfied by 
the $x$--coordinates of $P$, $Q$ and $R'$.
Therefore we must have the following factorization:
\[
\text{cubic polynomial} = c(x - x_1) (x - x_2) (x - x_3')
\]
where the left hand side is the cubic polynomial from equation (b).
Note that the cubic has leading coefficient 1, i.e. the coefficient
of $x^3$ is 1. Therefore
\[
\text{cubic polynomial} = (x - x_1) (x - x_2) (x - x_3')
\]
(i.e. $c = 1$).
Compute the coefficent of $x^2$ on the right of the above equation
in terms of $x_1, x_2, x_3'$.

\SOLUTION


\begin{align*}
0 &= a -{\lambda}^2
\end{align*}


\newpage

(Continuing Q1.)

(d) By equating the coefficients of $x^2$ of both sides of
the equation in (c), derive $x_3'$ in terms of the 
given data (i.e. the coefficient of $E$, the coefficients of $L$,
and the coordinates of $P$ and $Q$.) 

(The expression will contain $\lambda$. You need not substitute
$\lambda$ with its expression from (a).)

\SOLUTION


\begin{align*}
x_3' &= -x_2 - x_1 - a + {\lambda}^2
\end{align*}


\newpage

(e) In (d), you've derived $x_3'$ which is the $x$--coordinate of $R'$.
Note that $R'$ is on $L$. By substituting $x_3'$ in $L$, compute
the $y$--coordinate of $R'$, i.e. $y_3'$.

(The expression contains $\lambda$ and $\nu$. 
You need not substitute these with their expressions in (a).)

\SOLUTION

\input{q01e}

\newpage

(f) By the (geometric) definition of $P + Q$, 
the point $R$ is the reflection of $R'$ about the $x$--axis.
Using (e), state the coordinates of $R$, i.e. $x_3$ and $y_3$.

(The expression contains $\lambda$ and $\nu$. 
You need not substitute these with their expressions in (a).)

\SOLUTION


\begin{align*}
R &= ((-x_2 - x_1 -a + {\lambda}^2) , -(\lambda \cdot (-x_2 - x_1 -a + {\lambda}^2) + \nu))
\end{align*}



\newpage

As a summary, you can now state one case of 
your theorem on the addition formulas for $E$:

Let $E$ be the elliptic curve
\[
E: y^2 = x^3 + ax^2 + bx + c
\]
and 
\[
P = (x_1, y_1), \,\,\,\,\,
Q = (x_2, y_2), \,\,\,\,\,
x_1 \neq x_2
\]
Then 
\[
P + Q = (x_3, y_3)
\]
where
\begin{align*}
\lambda &= \frac{y_2 - y_1}{x_2 - x_1} \\
\nu &= y_1 - \biggr( \frac{y_2 - y_1}{x_2 - x_1} \cdot x_1 \biggl) \\
x_3 &= -x_2 - x_1 -a + {\lambda}^2 \\
y_3 &= \lambda \cdot (-x_2 - x_1 -a + {\lambda}^2) + \nu
\end{align*}

\newpage

ASIDE: 
Of course a good researcher {\it always} checks his/her work.
First the following points
\[
P = (3, 5)
\]
and 
\[
Q = \biggl( \frac{129}{10^2}, \frac{383}{10^3} \biggr)
\]
are on the elliptic curve
\[
E: y^2 = x^3 - 2
\]
Compute $P + Q$ using your formulas.
Check that the point is on $E$.
  
\newpage

Now for the next case.

Recall that  
\[
E: y^2 = x^3 + ax^2 + bx + c
\]
and
$P = (x_1, y_1)$ and $Q = (x_2, y_2)$ be points on $E$.
We want to derive the addition formulas for $R$ where $R = P + Q$.
Let $R = (x_3, y_3)$.

Q2. CASE: $x_1 = x_2$, $y_1 \neq y_2$.

What is $R$? [Just state it. No need to give the reason
because I already talked about it in class.]

\SOLUTION


(a)
\begin{align*}
\frac{1}{5 - 3x}
&= ??? \\
\end{align*}

ANSWER:
\boxed{???}
\qed


(b)

ANSWER:
\boxed{???}
\qed



\newpage

Now for the third case.
Again, let 
\[
E: y^2 = x^3 + ax^2 + bx + c
\]
and
let $P = (x_1, y_1)$ and $Q = (x_2, y_2)$ be points on $E$.
We want to derive the addition formulas for $R$ where $R = P + Q$.
Let $R = (x_3, y_3)$.

We now handle the case of $x_1 = x_2$ and $y_1 = y_2$, i.e. we 
want to compute $2P$.
We need to be careful since for this case the tangent line can be
vertical.
We first handle the case where the tangent line is not vertical.

Q3. CASE: $x_1 = x_2$, $y_1 = y_2$, $y_1 \neq 0$.
(Note that in this case $P = Q$ and $R = P + P = 2P$.)

(a) Let $L$ be the line
\[
L: y = \lambda x + \nu 
\] 
tangent to the curve $E$ at point $P$.
Derive $\lambda$ and $\nu$
terms of $x_1, y_1$.

\SOLUTION


\begin{align*}
\lambda &= \frac{3{x_1}^2 + 2 a x_1 + b}{2 y_1} \\
\\
\nu &= y_1 - \biggr( \frac{3{x_1}^2 + 2 a x_1 + b}{2 y_1} \cdot x_1 \biggl)\\
\end{align*}



\newpage

(Continuing Q3.)


(b) Let $R' = (x'_3, y'_3)$ be the third point of intersection of $E$ and $L$.
Therefore $P, Q (= P), R'$ must satisfy the equations of both $E$ and $L$.
\begin{align*}
y^2 &= x^3 + ax^2 + bx + c \\
y &= \lambda x + \nu
\end{align*}
Through substitution remove the variable $y$. 
You will obtain (of course) an equation in $x$ of the form:
\[
\text{cubic polynomial} = 0
\]
Derive and write down this equation.

Of course the solutions to the above equation
gives the three $x$--coordinates
of $P$, $Q (=P)$, and $R'$.

\SOLUTION

\input{q03b}

\newpage

(Continuing Q3.)

(c) We know that the equation from (b) is satisfied by 
the $x$--coordinates of $P$, $Q (= P)$ and $R'$.
Therefore we must have the following factorization:
\[
\text{cubic polynomial} = c(x - x_1) (x - x_1) (x - x_3')
\]
where the left hand side is the cubic polynomial from equation (b).
Note that the cubic has leading coefficient 1, i.e. the coefficient
of $x^3$ is 1. Therefore
\[
\text{cubic polynomial}  = (x - x_1) (x - x_1) (x - x_3')
\]
(i.e. $c = 1$).
Compute the coefficent of $x^2$ on the right of the above equation.

\SOLUTION

\input{q03c}

\newpage

(d) By equating the coefficients of $x^2$ of both sides of
the equation in (c), derive $x_3'$ in terms of the 
given data.

(The answer will contain $\lambda$. You need not replace
$\lambda$ with its expression from (a).)

\SOLUTION
 

\begin{align*}
x_3' &= -2x_1 - a + {\lambda}^2
\end{align*}


\newpage

(e) In (d), you've derived $x_3'$ which is the $x$--coordinate of $R'$.
Note that $R'$ is on $L$. By substituting $x_3'$ in $L$, compute
the $y$--coordinate of $R'$, i.e. $y_3'$

(The answer will contain $\lambda$ and $\nu$. You need not replace
$\lambda$ and $\nu$ with their expressions from (a).)
 
\SOLUTION


\begin{align*}
y_3' &= \lambda \cdot (-2x_1 - a + {\lambda}^2) + \nu
\end{align*}


\newpage

(f) By the (geometric) definition of $P + Q$ (when $Q = P$), 
the point $R$ is the reflection of $R'$ about the $x$--axis.
Using (e), state the coordinates of $R$, i.e. $x_3$ and $y_3$.

(The answer will contain $\lambda$ and $\nu$. You need not replace
$\lambda$ and $\nu$ with their expressions from (a).)



\begin{align*}
R &= ((-2x_1 - a + {\lambda}^2) , -(\lambda \cdot (-2x_1 - a + {\lambda}^2) + \nu))
\end{align*}


\newpage

As a summary, you can now state your theorem on 
addition formulas for 
finite points on our elliptic curve.

Let $E$ be the elliptic curve
\[
E: y^2 = x^3 + ax^2 + bx + c
\]
and 
\[
P = (x_1, y_1), \,\,\,\,\, y_1 \neq 0
\]
be a point on $E$.
Then 
\[
2P = P + P = (x_3, y_3)
\]
where
\begin{align*}
\lambda &= \frac{3{x_1}^2 + 2 a x_1 + b}{2 y_1}  \\
\nu &= y_1 - \biggr( \frac{3{x_1}^2 + 2 a x_1 + b}{2 y_1} \cdot x_1 \biggl) \\
x_3 &= -2x_1 - a + {\lambda}^2 \\
y_3 &= -(\lambda \cdot (-2x_1 - a + {\lambda}^2) + \nu) \\
\end{align*}

\newpage

ASIDE: 
Again, you should {\it always} checks your work.
First the following point
\[
P = (3, 5)
\]
is on the elliptic curve
\[
E: y^2 = x^3 - 2
\]
Compute $2P = P + P$ using your formulas and check that the point 
is on $E$.

\newpage

Now for the fourth case where we double a point with vertical
tangent line.
Again, let 
\[
E: y^2 = x^3 + ax^2 + bx + c
\]
and
let $P = (x_1, y_1)$ be a point on $E$ with $y_1 = 0$.

Q4. State $2P$ in this case.

[There's no need to explain since I have already mentioned this in class.]

\SOLUTION


(a)
\begin{align*}
\left(
\frac{1}{1 - 4x^2}
\right)^5
&= ??? \\
\end{align*}

ANSWER:
\boxed{???}
\qed


(b)

ANSWER:
\boxed{???}
\qed



\newpage

We have now handled all cases of adding finite points, including
cases where the points are distinct and the resulting tangle line is vertical
and the case of doubling a finite point with vertical tangent line.

The only cases left are additions where at least one point is the 
point at infinity ${\cal O}$.

All these cases are easy since by definition of the behavior of ${\cal O}$,
\[
P + {\cal O} = P = {\cal O} + P
\]
This includes the case of 
\[
{\cal O} + {\cal O} = {\cal O}\]

In terms of notation, instead of writing $(x,y)$ for finite points
and ${\cal O}$ for the point at infinity, 
we will also write finite points as
\[
(x:y:1)
\]
and the point at infinity as 
\[
(0:1:0)
\]

I don't want to go into details here, but we're basically viewing the
elliptic curve in a {\it projective} space. 
A 2-d space when placed in a corresponding 2-d projective space will
have 3 coordinates.

But even for computational reasons, the projective notation is helpful.
Why? Because for Python we can use a list $[x,y,1]$ for finite points
and $[0,1,0]$ to represent the point at infinity.

With the + now defined on the {\it projective} curve $E$, i.e. $E$
with the point at infinity $0$, one can prove that
the resulting points form a group with neutral element $0$.
In other words for $P, Q, R$ in $E$ (including the case where $P$
or $Q$ or $R$ is ${\cal O}$),
\begin{enumerate}
\item[$\bullet$] $P + Q$ is also a point of $E$ (closure)
\item[$\bullet$] $(P + Q) + R = P + (Q + R)$ (associativity)
\item[$\bullet$] There is some $P'$ such that 
$P + P' = {\cal O} = P' + P$.
We usually write the inverse of $P$ as $-P$ (inverse)
\item[$\bullet$] $P + {\cal O} = P = {\cal O} + P$ (neutral)
\end{enumerate}

Note that by definition, if the line through $P$ and $Q$ is vertical,
then 
\[
P + Q = {\cal O} = Q + P
\]
This implies that the inverse of $P$ is the point that is vertically
above or below $P$. 
Let $P = (x,y)$. 
Since we write the inverse of $P$ as $-P$, we have justs shown that
\[
-P = (x, -y)
\]
i.e.
\[
-(x, y) = (x, -y)
\]

\newpage

\newpage

Q5. You are already given the Python code for $\Z/N\Z$ 
(the ring of $\Z$ mod $N$.)

Using a Python list to represent points, i.e.
\begin{Verbatim}
    [x, y, 1]
\end{Verbatim}
for finite points and 
\begin{Verbatim}
    [0, 1, 0]
\end{Verbatim}
to present the point at infinity, implement a function to add points
on any elliptic curve 
\[
E: y^2 = x^3 + ax^2 + bx + c
\]
This is how your function should look like:
\begin{Verbatim}
    def add(E, N, P, Q):
        ...
\end{Verbatim}
The second parameter is a positive integer for the mod. 
For instance if we're interested in $\Z/23\Z$ points, then $N$ is 23.
The first parameter $E$ is a list of $a, b, c$ where the equation for 
$E$ is
\[
E: y^2 = x^3 + ax^2 + bx + c
\]
The values $a,b,c$ are $\Z/N\Z$ integers.
For instance when we want to study $\Z/23\Z$ points on 
\[
E: y^2 = x(x-1)(x+1) = x^3 - x = x^3 + 0x^2 + (-1)x + 0
\]
we have $a = 0, b = -1, c = 0$.
In this case the first parameter $E$ is
\begin{Verbatim}
    [ZN(0, 23), ZN(-1, 23), ZN(0, 23)]
\end{Verbatim}
For instance note that $P = (1,0)$ is a point on $E$.
Therefore to compute $2P$ I would call this:
\begin{Verbatim}
    N = 23
    E = [ZN(0, N), ZN(-1, N), ZN(0, N)]
    P = [ZN(1, N), ZN(0, N), ZN(1, N)]
    twoP = add(E, N, P, P)
\end{Verbatim}
Note that the function must of course work with the point at infinity.
For instance this should work:
\begin{Verbatim}
    N = 23
    E = [ZN(0, N), ZN(-1, N), ZN(0, N)]
    P = [ZN(1, N), ZN(0, N), ZN(1, N)]
    O = [ZN(0, N), ZN(1, N), ZN(0, N)]
    P_add_O = add(E, N, P, O)
\end{Verbatim}

You should have a folder containing this program which you should name
\verb!EC.py! and in the same folder you should have \verb!ZN.py!.
In your \verb!EC.py! you should have on the few lines the following:

\begin{Verbatim}[frame=single]

# Name: Brandy Poag
from ZN import *

def add(e, N, p, q):
    #print " e ", e
    
    a = e[0]
    b = e[1]
    c = e[2]
    #print "adding ", a, " ", b, " ", c
    x1= p[0]
    x2= q[0]
    y1= p[1]
    y2= q[1]
    #print "x1 ", x1, " y1 ", y1, " x2 ", x2, " y2 ", y2
    p_finite = p[2].data
    q_finite = q[2].data
    #print "p fin ", p_finite, "  q fin ", q_finite

    
    #check for infinite points
    if p_finite == 0:
        if q_finite == 0:
            return [ZN(0, N), ZN(1, N), ZN(0, N)]
        else:
            if (y2**2).data == (x2**3 + a*(x2**2) + b*x2 + c).data:
                return q
            else: return None
    elif q_finite == 0:
        if (y1**2).data == (x1**3 + a*(x1**2) + b*x1 + c).data:
            return p
        else: return None
        
    #make sure finite points are on the curve
    if not (y1**2 == x1**3 + a*(x1**2) + b*x1 + c) or \
       not (y2**2 == x2**3 + a*(x2**2) + b*x2 + c):
        print "print point not on curve"
        return None
    
    #handle finite points
    if not(x1 == x2):
        m = (y2 - y1) / (x2 - x1)
        #print "m  ", m
        c = y1 - (m * x1)
        #print "c  ", c
        x_3 = m**2 - x2 - x1 - a 
        #print "x_3  ", x_3
        y_3 = m * x_3 + c
        #print "y_3  ", y_3
        return [ZN(x_3.data, N), ZN(-(y_3.data), N), ZN(1, N)]
    elif not(y1 == y2):
        return [ZN(0, N), ZN(1, N), ZN(0, N)]
    elif not(y1 == 0):
        m = (ZN(3, N) * x1**2 + ZN(2, N) * a*x1 + b)/(ZN(2, N) * y1)
        #print "m  ", m
        c = y1 - (m * x1)
        #print "c  ", c
        x_3 = m**2 - ZN(2, N)*x1 - a 
        #print "x_3  ", x_3
        y_3 = m * x_3 + c
        #print "y_3  ", y_3
        return [ZN(x_3.data, N), ZN(-(y_3.data), N), ZN(1, N)]
    elif y1 == 0:
        return [ZN(0, N), ZN(1, N), ZN(0, N)]
    else:
        print "error with coordinates"
        return None

N = 7
E = [ZN(0, N), ZN(0, N), ZN(-2, N)]
#P = [ZN(1, N), ZN(0, N), ZN(1, N)]
#Q = [ZN(3, N), ZN(5, N), ZN(1, N)]
Q = [ZN(6, N), ZN(5, N), ZN(1, N)]
P = [ZN(3, N), ZN(2, N), ZN(1, N)]
O = [ZN(0, N), ZN(1, N), ZN(0, N)]

print "p: (", P[0].data,", ", P[1].data, ")"
print "q: (", Q[0].data,", ", Q[1].data, ")"
print "O: (", O[0].data,", ", O[1].data, ")"

print "P_add_P:"
P_add_P = add(E, N, P, P)
if P_add_P != None: print "[", P_add_P[0].data, " , ", 
	P_add_P[1].data, " , ", P_add_P[2].data, "]"
else: print "no point returned"

print

print "Q_add_Q:"
Q_add_Q = add(E, N, Q, Q)
if Q_add_Q != None: print "[", Q_add_Q[0].data, " , ", 
	Q_add_Q[1].data, " , ", Q_add_Q[2].data, "]"
else: print "no point returned"

print

print "P_add_Q:"
P_add_Q = add(E, N, P, Q)
if P_add_Q != None: print "[", P_add_Q[0].data, " , ", 
	P_add_Q[1].data, " , ", P_add_Q[2].data, "]"
else: print "no point returned"

print

print "Q_add_P:"
Q_add_P = add(E, N, Q, P)
if Q_add_P != None: print "[", Q_add_P[0].data, " , ", 
	Q_add_P[1].data, " , ", Q_add_P[2].data, "]"
else: print "no point returned"

print

print "P_add_O:"
P_add_O = add(E, N, P, O)
if P_add_O != None: print "[", P_add_O[0].data, " , ", 
	P_add_O[1].data, " , ", P_add_O[2].data, "]"
else: print "no point returned"

print

print "O_add_P:"
O_add_P = add(E, N, O, P)
if O_add_P != None: print "[", O_add_P[0].data, " , ",
	 O_add_P[1].data, " , ", O_add_P[2].data, "]"
else: print "no point returned"

print

print "O_add_O:"
O_add_O = add(E, N, O, O)
if O_add_O != None: print "[", O_add_O[0].data, " , ", 
	O_add_O[1].data, " , ", O_add_O[2].data, "]"
else: print "no point returned"

print

print "Q_add_O:"
Q_add_O = add(E, N, Q, O)
if Q_add_O != None: print "[", Q_add_O[0].data, " , ", 
	Q_add_O[1].data, " , ", Q_add_O[2].data, "]"
else: print "no point returned"

print

print "O_add_Q:"
O_add_Q = add(E, N, O, Q)
if O_add_Q != None: print "[", O_add_Q[0].data, " , ", 
	O_add_Q[1].data, " , ", O_add_Q[2].data, "]"
else: print "no point returned"
\end{Verbatim}
\end{document}
