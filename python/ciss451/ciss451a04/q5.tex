
\lq\lq $a \equiv b \equiv c \pmod{n}$ means
\begin{align*}
a &\equiv b \pmod{n} \\
n \mid a-b && \text{by CON1} \\
nx \equiv a-b && \text{by DIV1} \\
nx - -b \equiv a && \text{by Thm Brandy1A}\\
nx + b \equiv a && \text{by THM4A}\\
\\
\text{ and } \\
\\
b &\equiv c \pmod{n} \\
n \mid b-c && \text{by CON1} \\
ny \equiv b-c && \text{by DIV1} \\
nx - b \equiv -c && \text{by Thm Brandy1B}\\
\\
(nx + b) + (ny + -b) \equiv a + (-c) \\
nx + (b + ny + -b) \equiv a + (-c) && \text{by RING2}\\
nx + (b + ny) + -b \equiv a + (-c) && \text{by RING3}\\
nx + (ny + b) + -b \equiv a + (-c) && \text{by RING10}\\
(nx + ny) + (b + -b) \equiv a + (-c) && \text{by RING2 and RING3}\\
(nx + ny) + (0) \equiv a + (-c) && \text{by RING8}\\
(nx + ny) \equiv a + (-c) && \text{by RING4}\\
n(x + y) \equiv a + (-c) && \text{by RING19}\\
n(x + y) \equiv a - c && \text{by RING24}\\
n \mid a - c  && \text{by DIV2}\\
\THEREFORE a \equiv c \pmod{n}  && \text{by CON2}\\
\end{align*}
