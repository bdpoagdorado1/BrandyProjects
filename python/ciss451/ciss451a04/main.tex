\newcommand\COURSENAME{Cryptography and Computer Security}
\newcommand\COURSESHORTNAME{Cryptography}
\newcommand\COURSENUMBER{CISS451/MATH451}
\newcommand\TITLE{Assignment 4}
\newcommand\AUTHOR{BRANDY POAG-DORADO}

\documentclass[a4paper,12pt]{article}
\usepackage{assignment}
\definecolor{red}{rgb}{1,0,0}
\newcommand\redtext[1]{\textcolor{red}{#1}}
\begin{document}
\topmatter

This is an assignment on the congruence notation.

\newpage

Here are some facts about $(\Z, +, \cdot, 0, 1)$.
In the following $x,y,z$ are integers, i.e. $x,y,z \in \Z$.

\newcommand\LISTMARGIN{1in}
\begin{enumerate}[leftmargin=\LISTMARGIN]
\item[RING1:] $x,y \in \Z$ can be replaced by $x+y \in \Z$
\item[RING2:] $(x + y) + z$ can be replaced by $x + (y + z)$
\item[RING3:] $x + (y + z)$ can be replaced by $(x + y) + z$
\item[RING4:] $x + 0$ can be replaced by $x$
\item[RING5:] $x$ can be replaced by $x + 0$
\item[RING6:] $0 + x$ can be replaced by $x$
\item[RING7:] $x$ can be replaced by $0 + x$
\item[RING8:] There is some $-x \in \Z$ 
              such that $x + (-x)$ can be replaced by 0.
\item[RING8B:] There is some $-x \in \Z$ 
              such that 0 can be replaced by $x + (-x)$.
\item[RING9:] There is some $-x \in \Z$ 
              such that $(-x) + x$ can be replaced by 0.
\item[RING9B:] There is some $-x \in \Z$ 
              such that $0$ can be replaced by $(-x) + x$.
\item[RING10:] $x+y$ can be replaced by $y + x$.
\item[RING11:] $x, y \in \Z$ can be replaced by $xy \in \Z$
\item[RING12:] $(xy)z$ can be replaced by $x(yz)$
\item[RING13:] $x(yz)$ can be replaced by $(xy)z$
\item[RING14:] $x1$ can be replaced by $x$
\item[RING15:] $x$ can be replaced by $x1$
\item[RING16:] $1x$ can be replaced by $x$
\item[RING17:] $x$ can be replaced by $1x$
\item[RING18:] $x(y + z)$ can be replaced by $xy + xz$
\item[RING19:] $xy + xz$ can be replaced by $x(y + z)$
\item[RING20:] $(y+z)x$ can be replaced $yx + zx$
\item[RING21:] $yx+zx$ can be replaced $(y + z)x$
\item[RING22:] $xy$ can be replaced by $yx$
\item[RING23:] $xy=0$ can be replaced by [$x=0$ or $y=0$].
\end{enumerate}

The following are a notational rewrite rules for subtraction
(i.e. the following are definitions and not axioms):
\begin{enumerate}[leftmargin=\LISTMARGIN]
\item[RING24:] $x - y$ can be replaced by $x + (-y)$
\item[RING25:] $x + (-y)$ can be replaced by $x - y$
\end{enumerate}

The following are from Theorems 1 and 2 proven in Assignment 1 (i.e.
$0x = 0 = x0$)
\begin{enumerate}[leftmargin=\LISTMARGIN]
\item[TH1A:] $0x$ can be replaced by $0$
\item[TH1B:] $0$ can be replaced by $0x$
\item[TH1C:] $0$ can be replaced by $x0$
\item[TH1D:] $x0$ can be replaced by $0$
\item[TH1E:] $0x$ can be replaced by $x0$
\item[TH1F:] $x0$ can be replaced by $0x$
\end{enumerate}

Here are some facts from Theorem 3 (i.e. $x+y = 0 \implies y = -x$ and
$y + x \implies y=-x$):
\begin{enumerate}[leftmargin=\LISTMARGIN]
\item[TH3A:] If $x+y=0$, then $y$ can be rewritten as $-x$.
\item[TH3B:] If $x+y=0$, then $-x$ can be rewritten as $y$.
\item[TH3C:] If $y+x=0$, then $y$ can be rewritten as $-x$.
\item[TH3D:] If $y+x=0$, then $-x$ can be rewritten as $y$.
\end{enumerate}

Here are some facts from Theorem 4 from Assignment 1:
\begin{enumerate}[leftmargin=\LISTMARGIN]
\item[TH4A:] $-(-x)$ can be rewritten as $x$.
\item[TH4B:] $x$ can be rewritten as $-(-x)$.
\end{enumerate}

Here's Theorem 5 from Assignment 2 (i.e. $-1 \cdot x = -x = x \cdot (-1)$):
\begin{enumerate}[leftmargin=\LISTMARGIN]
\item[TH5A:] $-1 \cdot x$ can be replaced by $-x$
\item[TH5B:] $-x$ can be replaced by $-1 \cdot x$
\item[TH5C:] $-x$ can be replaced by $x \cdot (-1)$
\item[TH5D:] $x \cdot (-1)$ can be replaced by $-x$ 
\end{enumerate}

Here's Theorem 6 from Assignment 2 (i.e. $(-y)x = -(yx) = y(-x)$):
\begin{enumerate}[leftmargin=\LISTMARGIN]
\item[TH6A:] $(-y)x$ can be replaced by $-(yx)$
\item[TH6B:] $-(yx)$ can be replaced by $(-y)x$
\item[TH6C:] $-(yx)$ can be replaced by $y(-x)$
\item[TH6D:] $y(-x)$ can be replaced by $-(yx)$
\item[TH6E:] $(-y)x$ can be replaced by $y(-x)$
\item[TH6F:] $y(-x)$ can be replaced by $(-y)x$ 
\end{enumerate}

Here's Theorem 7 from Assignment 2 (i.e. $(-x)(-y) = xy$ and $(-1)(-1)=1$):
\begin{enumerate}[leftmargin=\LISTMARGIN]
\item[TH7A:] $(-x)(-y)$ can be replaced by $xy$
\item[TH7B:] $xy$ can be replaced by $(-x)(-y)$
\item[TH7C:] $(-1)(-1)$ can be replaced by $1$
\item[TH7D:] $1$ can be replaced by $(-1)(-1)$
\end{enumerate}

Here's Theorem 8 ($0 = -0$):
\begin{enumerate}[leftmargin=\LISTMARGIN]
\item[TH8A:] $0$ can be replaced by $-0$
\item[TH8B:] $-0$ can be replaced by $0$
\end{enumerate}

Here's Theorem 9 (the additive cancellation property):
\begin{enumerate}[leftmargin=\LISTMARGIN]
\item[TH9A:] If $a + x = a + y$, then $x$ can be replaced by $y$.
\item[TH9B:] If $x + a = y + a$, then $x$ can be replaced by $y$.
\end{enumerate}

Here's Theorem 10 (the multiplicative cancellation property):
\begin{enumerate}[leftmargin=\LISTMARGIN]
\item[TH9A:] If $ax = ay, a \neq 0$, then $x$ can be replaced by $y$.
\item[TH9B:] If $xa = ya, a \neq 0$, then $x$ can be replaced by $y$.
\end{enumerate}

\newpage

Now for stuff on divisibility.

Here are the "rewrite rules'' for the definition of "divisibility'':
\begin{enumerate}[leftmargin=\LISTMARGIN]
\item[DIV1:] $d \mid a$ can be replaced by [$dx = a$ for some $x \in \Z$]
\item[DIV2:] [$dx = a$ for some $x \in \Z$] can be replaced by $d \mid a$ 
\end{enumerate}
which are really the same as:
\begin{enumerate}[leftmargin=\LISTMARGIN]
\item[DIV1:] $d \mid a$ can be replaced by [$\exists x \in \Z (dx = a)$]
\item[DIV2:] [$\exists x \in \Z (dx = a)]$ can be replaced by $d \mid a$ 
\end{enumerate}

Here's Theorem 11 and 12:
\begin{enumerate}[leftmargin=\LISTMARGIN]
\item[THM12A:] $a \mid a$. 
\item[THM12B:] If $a \mid b$ and $b \mid c$, then $a \mid c$. 
\end{enumerate}

Here's Theorem 13 (a whole mouthful of it ...)
\begin{enumerate}[leftmargin=\LISTMARGIN]
\item[THM13A:]$1 \mid a$.
\item[THM13B:]$-1 \mid a$.
\item[THM13C:] If $d \mid a$, then $-d \mid a$
\item[THM13D:] If $d \mid a$, then $d \mid ax$
\item[THM13E:] If $d \mid a$ and $d \mid b$, then $d \mid (a + b)$.
\item[THM13F:] Given integers $x$ and $y$, 
           if $d \mid a$ and $d \mid b$, then $d \mid (ax + by)$.
\end{enumerate}

\newpage

Theorem Foil:

Thm Foil A.)\\
$(w + x) (y + z)$ can be replaced by $(wy + wz) + (xy + xz)$
\begin{align*}
(w + x) (y + z)
&= (w (y + z) + x (y + z)) && \text{by RING20} \\
&=(wy + wz) + (xy + xz) && \text{by RING18} \\
\end{align*}

Thm Foil B.)\\
$(w + x) (y + z)$ can be replaced by $(wy + xy) + (wz + xz)$
\begin{align*}
(w + x) (y + z)
&=((w + x)y + (w + x)z) && \text{by RING18} \\
&=(wy + xy) + (wz + xz) && \text{by RING18} \\
\end{align*}


Theorem Brandy1:

Thm Brandy1A.)\\
$a = (b + c)$ can be replaced by $a - c = b$
\begin{align*}
a &= (b + c) \\
a + -c &= (b + c) + -c \\
a + -c &= b + (c + -c) && \text{by RING2} \\
a + -c &= b + 0 && \text{by RING8} \\
a + -c &= b && \text{by RING4} \\
a - c &= b && \text{by RING25} \\
\end{align*}

Thm Brandy1B.)\\
$a = (b + c)$ can be replaced by $a - b = c$
\begin{align*}
a &= (b + c) \\
a + -b &= -b + (b + c) \\
a + -b &= (-b + b) + c && \text{by RING3} \\
a + -b &= 0 + c && \text{by RING9} \\
a + -b &= c && \text{by RING4} \\
a - b &= c && \text{by RING25} \\
\end{align*}

Thm Brandy1C.)\\
$(a + b) = c$ can be replaced by $b = c - a$
\begin{align*}
(a + b) &= c \\
-a + (a + b) &= c + -a \\
(-a + a) + b &= c + -a && \text{by RING3} \\
0 + b &= c + -a && \text{by RING9} \\
b &= c + -a && \text{by RING4} \\
b &= c - a && \text{by RING25} \\
\end{align*}

Thm Brandy1D.)\\
$(a + b) = c$ can be replaced by $a = c - b$
\begin{align*}
(a + b) &= c \\
(a + b) + -b &= c + -b \\
a + (b + -b) &= c + -b && \text{by RING2} \\
a + 0 &= c + -b && \text{by RING8} \\
a &= c + -b && \text{by RING4} \\
a &= c - b && \text{by RING25} \\
\end{align*}

\newpage

Here are some questions on GCD.

For the following we will assume that you have access to tables or a 
computer that can compute only integer quotients and remainders, 
i.e., that you have access to a Euclidean property machine.
For instance, if you're given 100 and 23 and you want want to find $q,r$
such that
\[
100 = 23 \cdot q + r, \,\,\, 0 \leq r < 23
\]
you can of course compute $q$ and $r$ using C++ like this:
\begin{Verbatim}[frame=single]
std::cout << 100 / 23 << ',' << 100 % 23 << '\n';
\end{Verbatim}
You can do this in Python:
\begin{Verbatim}[frame=single]
print divmod(100, 23)
\end{Verbatim}

\newpage

Q1. \mbox{}
\begin{enumerate}[topsep=0in]
\item[(a)] Write down the prime factorization of 123556.
\item[(b)] Write down the prime factorization of 5436.
\item[(c)] Compute $\gcd(123556, 5436)$ using the above prime factorizations.
\end{enumerate}

Here's an example of how you must present your prime factorization:
\[
300 = 2^2 \cdot 3^1 \cdot 5^2
\]

\SOLUTION


a) $123556 = 2^2 \cdot 17^1 \cdot 23^1 \cdot 79^1$

b) $5436 = 2^2 \cdot 3^2 \cdot 151^1$

c) $gcd(123556, 5436) = 2^2 = 4$



\newpage

Q2. 
Write a list of Euclidean computations to compute the 
$\gcd(123556, 5436)$.
The last remainder must be 0.

(If you don't know what I'm talking about, then you have not read
the notes I gave you. Read it.)

\SOLUTION


$123556 = 5436 \cdot 22 + 3964$ \\
$5436 = 3964 \cdot 1 + 1472$ \\
$3964 = 1472 \cdot 2 + 1020$ \\
$1472 = 1020 \cdot 1 + 425$ \\
$1020 = 425 \cdot 2 + 116$ \\
$425 = 116 \cdot 3 + 104$ \\
$116 = 104 \cdot 1 + 12$ \\
$104 = 12 \cdot 8 + 8$ \\
$12 = 8 \cdot 1 + 4$ \\
$8 = 4 \cdot 2 + 0$ \\

So 4 is the gcd(123556, 5436)


\newpage




Q3.
Find integers $x$ and $y$ such that 
\[
123556x + 5436y = \gcd(123556, 5436)
\]
You should begin with labeling all the Euclidean computations
from Q2 and then perform substitutions.
Make sure you indicate very clearly which equation you're using.
Here's a reminder on how to label equations:
\[
1 + 1 = 2 \tag{1}
\]

\SOLUTION


$(12 \cdot 1) + (8 \cdot -1) = 4$ \\
$(12 \cdot 1) + ((104 \cdot 1) + (12 \cdot -8)) -1 = 4$ \\
$(12 \cdot 9) + (104 \cdot -1) = 4$ \\
$((116 \cdot 1) + (104 \cdot -1)) 9 + (104 \cdot -1) = 4$ \\
$(116 \cdot 9) + (104 \cdot -10) = 4$ \\
$(116 \cdot 9) + ((425 \cdot 1) + (116 \cdot -3)) -10 = 4$ \\
$(116 \cdot 39) + (425 \cdot - 10) = 4$ \\
$((1020 \cdot 1) + (425 \cdot -2)) 39 + (425 \cdot - 10) = 4$ \\
$(1020 \cdot 39) + (425 \cdot - 88) = 4$ \\
$(1020 \cdot 39) + ((1472 \cdot 1) + (1020 \cdot -1))-88 = 4$ \\
$(1020 \cdot 127) + (1472 \cdot -88) = 4$ \\
$((3964 \cdot 1) + (1472 \cdot -2)) 127 + (1472 \cdot -88) = 4$ \\
$(3964 \cdot 127) + (1472 \cdot -342) = 4$ \\
$(3964 \cdot 127) + ((5436 \cdot 1) + (3964 \cdot -1)) -342 = 4$ \\
$(3964 \cdot 469) + (5436 \cdot -342) = 4$ \\
$((123556 \cdot 1) + (5436 \cdot -22))469 + (5436 \cdot -342) = 4$ \\
$(123556 \cdot 469) + (5436 \cdot -10660) = 4$ \\

$123556x + 5436y = gcd(123556, 5436) = 4$

So $x = 469$ and $y = -10660$.




\newpage

Q4. Show that if you have the following equations:
\begin{align*}
c_1 r_0 + d_1 r_1 &= r_3 \tag{1} \\
c_2 r_0 + d_2 r_1 &= r_4 \tag{2} \\
r_3 + (-q_4) r_4  &= r_5 \tag{3}
\end{align*}
Show that 
\[
(c_1 - q_4c_2) r_0 + (d_1 - q_4d_2)r_1 = r_5
\]

For this question, you need not quote the RING properties/axioms or
theorems, i.e., just treat this as a \lq\lq normal'' algebra 
manipulation problem.

\SOLUTION


$c_1 r_0 + d_1 r_1 + (-q_4) r_4  = r_5$ \\
$(c_1 r_0 + d_1 r_1) + (-q_4) (c_2 r_0 + d_2 r_1)  = r_5$ \\
$(c_1 r_0 + d_1 r_1) + ((-q_4)c_2 r_0 + (-q_4)d_2 r_1)  = r_5$ \\
$(c_1 r_0 - q_4c_2 r_0) + (d_1 r_1 - q_4 d_2 r_1) = r_5$ \\
$\THEREFORE (c_1 - q_4c_2)r_0 + (d_1 - q_4 d_2)r_1 = r_5$




\newpage
Now for the rewrite rules for the congruence notation for modular
arithmetic:

\begin{enumerate}[leftmargin=\LISTMARGIN]
\item[CON1:] $a \equiv b \pmod{n}$ can be replaced by $n \mid (a-b)$
\item[CON2:] $n \mid (a-b)$ can be replaced by $a \equiv b \pmod{n}$
\end{enumerate}

If $a \equiv b \pmod{n}$ then we say \lq\lq $a$ is congruent to
$b$ mod $n$''.

\newpage

\setcounter{thm}{13}

\begin{thm} Let $a,n$ be integers with $n > 0$.
$a \equiv 0 \pmod{n}$ iff $n \mid a$
\end{thm}

This is easy to prove.
I'll do this one for you quickly.
If $a \equiv 0 \pmod{n}$, then by definition
(make sure you check that I'm not lying!), 
$n$ divides $a - 0$, i.e. $n$ divides $a$.
And if $n$ divides $a$, then $n$ divides $a - 0$,
and therefore $a \equiv 0 \pmod{n}$.

You might want to write it out in full.
Note that I used the fact that $a$ is the same as $a - 0$.
(Is this true? Are you sure? Don't say I didn't warn you.)

\newpage

\begin{thm} \mbox{}
\begin{enumerate}[topsep=0in]
\item[(a)] $a \equiv a \pmod{n}$
\item[(b)] $a \equiv b \pmod{n}$ $\implies$ $b \equiv a \pmod{n}$
\item[(c)] $a \equiv b \equiv c \pmod{n}$ $\implies$ $a \equiv c \pmod{n}$
\end{enumerate}
\end{thm}

These are called (respectively) the reflexive, symmetric, and 
transitive properties of the congruence relation.
\lq\lq $a \equiv b \equiv c \pmod{n}$'' means
\begin{align*}
a &\equiv b \pmod{n} \\
b &\equiv c \pmod{n}
\end{align*}

By the way, there are lots of relations which are reflexive, symmetric,
and transitive.
For example, if the universe is the set of all lines in 2-d space,
then the relation \lq\lq is parallel to'' is reflexive,
symmetric, and transitive.
Is $<$ on real numbers reflexive? symmetric? transitive?

How about the \lq\lq is married to'' relation on people?

How about \lq\lq feeds on'' relations in the food web?


\newpage

Q5. Prove Theorem 15(c).

\SOLUTION


\lq\lq $a \equiv b \equiv c \pmod{n}$ means
\begin{align*}
a &\equiv b \pmod{n} \\
n \mid a-b && \text{by CON1} \\
nx \equiv a-b && \text{by DIV1} \\
nx - -b \equiv a && \text{by Thm Brandy1A}\\
nx + b \equiv a && \text{by THM4A}\\
\\
\text{ and } \\
\\
b &\equiv c \pmod{n} \\
n \mid b-c && \text{by CON1} \\
ny \equiv b-c && \text{by DIV1} \\
nx - b \equiv -c && \text{by Thm Brandy1B}\\
\\
(nx + b) + (ny + -b) \equiv a + (-c) \\
nx + (b + ny + -b) \equiv a + (-c) && \text{by RING2}\\
nx + (b + ny) + -b \equiv a + (-c) && \text{by RING3}\\
nx + (ny + b) + -b \equiv a + (-c) && \text{by RING10}\\
(nx + ny) + (b + -b) \equiv a + (-c) && \text{by RING2 and RING3}\\
(nx + ny) + (0) \equiv a + (-c) && \text{by RING8}\\
(nx + ny) \equiv a + (-c) && \text{by RING4}\\
n(x + y) \equiv a + (-c) && \text{by RING19}\\
n(x + y) \equiv a - c && \text{by RING24}\\
n \mid a - c  && \text{by DIV2}\\
\THEREFORE a \equiv c \pmod{n}  && \text{by CON2}\\
\end{align*}


\newpage

\begin{thm} Let $n > 0$ and $a,b,c,d$ be integers.
If
\begin{align*}
a &\equiv b \pmod{n} \\
c &\equiv d \pmod{n}
\end{align*}
then
\[
a+c \equiv b+d \pmod{n} \\
\]
\end{thm}

\newpage 

Q6. Prove Theorem 16.

\SOLUTION



From $a \equiv b \pmod{n}$ and $c \equiv d \pmod{n}$ we have:

\begin{align*}
n \mid (a-b) && \text{by CON1} \\
\text{ and } \\
n \mid (c-d) && \text{by CON1} \\
nx \equiv (a-b) && \text{by DIV1} \\
\text{ and } \\
ny \equiv (c-d) && \text{by DIV1} \\
nx + ny \equiv (a-b) + (c-d) \\
\THEREFORE n(x + y) \equiv (a-b) + (c-d) && \text{by RING19} \\
\THEREFORE n(x + y) \equiv (a + (-b)) + (c + (-d)) && \text{by RING24} \\
\THEREFORE n(x + y) \equiv a + (-b + (c + -d)) && \text{by RING2} \\
\THEREFORE n(x + y) \equiv a + ((-b + c) + -d) && \text{by RING3} \\
\THEREFORE n(x + y) \equiv a + ((c + -b) + -d) && \text{by RING10} \\
\THEREFORE n(x + y) \equiv a + (c + (-b + -d)) && \text{by RING2} \\
\THEREFORE n(x + y) \equiv (a + c) + (-b + -d) && \text{by RING3} \\
\THEREFORE n(x + y) \equiv (a+c) + (-b-d) && \text{by RING25} \\
\THEREFORE n(x + y) \equiv (a+c) - (b+d) && \text{by RING24} \\
\THEREFORE n \mid (a+c) - (b+d)  && \text{ by DIV2}\\
\THEREFORE a+c \equiv b+d  && \text{by CON2}
\end{align*}

\newpage



\begin{thm}
Let $n > 0$ and $a,b,c,d$ be integers.
Prove that if
\begin{align*}
a &\equiv b \pmod{n} \\
c &\equiv d \pmod{n} \\
\end{align*}
then
\[
ac \equiv bd \pmod{n} \\
\]
\end{thm}


\newpage

Q7. Prove Theorem 17.


\SOLUTION

\begin{align*}
n \mid a + -b && \text{by CON1} \\
nx &= a + -b && \text{by DIV1} \\
nx - -b &= a && \text{by Thm Brandy1B} \\
nx + b &= a && \text{by TH4A} \\
\text{ and } \\
n \mid c + -d && \text{by CON1} \\
ny &= c + -d && \text{by DIV1} \\
ny - -d &= c && \text{by Thm Brandy1A} \\
ny + d &= c && \text{by TH4A} \\
\\
(nx + b) \cdot (ny + d) &= ac\\
n^2xy + nxd + nyb + bd &= ac && \text{by Thm Foil} \\
n^2xy + nxd + nyb &= ac + -bd && \text{by Thm Brandy1D} \\
n(nxy + xd + yb) &= ac + -bd && \text{by RING19} \\
n(nxy + xd + yb) &= ac - bd && \text{by RING25} \\
n \mid ac - bd && \text{by DIV2} \\
\THEREFORE ac \equiv bd && \text{by CON2} \\
\end{align*}

\newpage


Q8. \mbox{}
\begin{enumerate}[topsep=0in]
\item[(a)]
What is the ones digit (or the unit digit) of
\[
1^{100} \cdot 2^{100} \cdot 3^{100} \cdot 4^{100} \cdot 5^{100}
\]
\item[(b)] What about this one:
\[
5^{1000} \cdot 11^{1000} \cdot 13^{1000} \cdot 17^{1000} \cdot 19^{1000}
\]
\item[(c)]
And this:
\[
23^{10000} + 29^{10000} + 31^{10000} + 37^{10000} +
43^{10000}
\]
\item[(d)] ... one last one [extra credit]:
\[
{{{{{123^{234}}^{345}}^{456}}^{567}}^{678}}^{789}
\]
\end{enumerate}

Justify all your work. Writing down the final answer give you a zero.
Nada. Zilch. 
I'm looking for a creative using of mathematical facts rather than trying to 
use a supercomputer to do the number crunching for you.

\SOLUTION


a)\\
$1^{100} = 1$ \\

$2^{10} = 1024 \equiv 4 \pmod{10}$ \\
$2^{100} = 4^{10} = 1048576 \equiv 6 \pmod{10}$\\

$3^{4} = 81 \equiv 1 \pmod{10}$\\
$3^{100} = {3^{4}}^{25} = 1^{25} \equiv 1 \pmod{10}$\\

$4^{1} = 4$ and $4^{2} = 16 \equiv 6 \pmod{10}$ this continues so evens end in 6 and odds end in 4\\
$4^{100} \equiv 6 \pmod{10}$\\

$5^{1} = 5 \equiv 5 \pmod{10}$\\ this continues all these end in 5
$5^{100} \equiv 5 \pmod{10}$\\

Altogether $1 \cdot 6 \cdot 1 \cdot 6 \cdot 5 = 180 \equiv 0 \pmod{10}$\\
So the last digit is a zero.

b)\\
$5^{1000} \equiv 5 \pmod{10}$\\

$11^{1} = 11 \equiv 1 \pmod{10}$ \\
$11^{1000} \equiv 1 \pmod{10}$\\

$13^{1} \equiv 3$ so 13 behaves lik 3 $3^{4} = 81 \equiv 1 \pmod{10}$\\
$13^{1000} \equiv 1 \pmod{10}$\\

$17^{1} \equiv 7$ so 17 behaves lik 7 $7^{4} = 2401 \equiv 1 \pmod{10}$\\
$17^{1000} \equiv 1 \pmod{10}$\\

$19^{1} \equiv 9$ so 19 behaves lik 9 $9^{2} = 81 \equiv 1 \pmod{10}$\\
$19^{1000} \equiv 1 \pmod{10}$\\

Altogether $5 \cdot 1 \cdot 1 \cdot 1 \cdot 1 = 5 \equiv 5 \pmod{10}$\\
So the last digit is a 5.

c)\\
$23^{1} \equiv 3$ so 23 behaves lik 3 $3^{4} = 81 \equiv 1 \pmod{10}$\\
$23^{1000} \equiv 1 \pmod{10}$\\

$29^{1} \equiv 9$ so 29 behaves lik 9 $9^{2} = 81 \equiv 1 \pmod{10}$\\
$29^{1000} \equiv 1 \pmod{10}$\\

$31^{1} \equiv 1$ so 31 behaves like 1\\
$31^{1000} \equiv 1 \pmod{10}$\\

$37^{1} \equiv 7$ so 37 behaves like 7 $7^{4} = 2401 \equiv 1 \pmod{10}$\\
$37^{1000} \equiv 1 \pmod{10}$\\

$43^{1} \equiv 3$ so 43 behaves lik 3 $3^{4} = 81 \equiv 1 \pmod{10}$\\
$43^{1000} \equiv 1 \pmod{10}$\\


Altogether $1 + 1 + 1 + 1 + 1 = 5 \equiv 5 \pmod{10}$\\
So the last digit is a 5.

d)\\
${123^{234}} = {123^{4}}^{58} \cdot 123^{2}$\\
$1^{58} \cdot 123^{2} = 123^{2}$\\
$123 \equiv 3 \pmod{10}$ so $123^{2} \equiv 3^{2} = 9$\\
$9^{345} \equiv 9$ because 345 is odd \\
$9^{456} \equiv 1$ because 456 is even \\
${1^{678}}^{789} \equiv 1 \pmod{10}$\\



So ${{{{{123^{234}}^{345}}^{456}}^{567}}^{678}}^{789} \equiv 1 \pmod{10}$\\




\end{document}
