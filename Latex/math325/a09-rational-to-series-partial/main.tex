\newcommand\COURSENAME{Discrete Math 2}
\newcommand\COURSESHORTNAME{Discrete Math 2}
\newcommand\COURSENUMBER{MATH325}
\newcommand\TITLE{Assignment 9}
\newcommand\AUTHOR{Dr Yihsiang Liow}

\documentclass[a4paper,12pt]{article}
\usepackage{assignment}
\definecolor{red}{rgb}{1,0,0}
\newcommand\redtext[1]{\textcolor{red}{#1}}
\begin{document}
\topmatter

This assignment involves converting rational functions to 
power series and partial fractions.

The two most important formulas for converting rational functions to 
power series are
\[
\frac{1}{1-x} = \sum_{n=0}^\infty x^n
\]
and
\[
\left(
\frac{1}{1-x}
\right)^k
= \sum_{n=0}^\infty \binom{k+n-1}{n} x^n
\]
(Well ... actually the second formula include the first ... $k=1$ in
the second gives you the first.)
In order to use both, remember that $x$ is a variable.
For instance you can think of the first formula as:
\[
\frac{1}{1- \operatorname{BLAH}} = \sum_{n=0}^\infty \operatorname{BLAH}^n
\]
So for instance this is also true
\[
\frac{1}{1- 12345x} 
= \sum_{n=0}^\infty (12345x)^n
= \sum_{n=0}^\infty 12345^n x^n
\]
Remember that if you want to test the equality above
with a specific value for $x$, you can only use a 
value for $x$ such that $|12345x| < 1$, i.e., $|x| < 1/12345$.
If you go outside this range, the power series will very likely blow up 
in your face.
Also,
\[
\left(
\frac{1}{1-123x^{567}}
\right)^{999}
= \sum_{n=0}^\infty \binom{999+n-1}{n} (123x^{567})^n
= \sum_{n=0}^\infty \binom{999+n-1}{n} 123^n x^{567n}
\]

Note however that the two 1's must be 1's:
\[
\frac{\underline{1}}{\underline{1}-x} = \sum_{n=0}^\infty x^n
\]
As shown in class if they are not 1's, then you just ... well ... make them 1:
\[
\frac{111}{222-333x} 
= 111 \frac{1}{222-333x} 
= \frac{111}{222} \frac{1}{1 -333x/222} 
\]

In questions where you are asked to read a coefficient, if the value is too
big, you can simply tidy up the expression and leave it as it is, i.e.,
you need not evaluate the expression to get a value. 
You can leave huge powers, binomial coefficients, etc.~alone.


\newpage
Q1. (a) Rewrite
\[
\frac{1}{1 - 2x}
\]
as a power series. 

(b) What is the coefficient of $x^{1000}$?

\SOLUTION


(a)
\begin{align*}
\frac{1}{1 - 2x}
&= ??? \\
\end{align*}

ANSWER:
\boxed{???}
\qed


(b)

ANSWER:
\boxed{???}
\qed


\newpage




Q2.
(a) Rewrite
\[
\frac{1}{5x - 3}
\]
as a power series. 

[HINT: Remember to make the above like $\frac{1}{1 - \operatorname{BLAH}}$.]

(b) What is the coefficient of $x^{1000}$?

\SOLUTION


(a)
\begin{align*}
\frac{1}{5 - 3x}
&= ??? \\
\end{align*}

ANSWER:
\boxed{???}
\qed


(b)

ANSWER:
\boxed{???}
\qed


\newpage


Q3.
(a) Rewrite
\[
\left( 
\frac{1}{2x - 3}
\right)^5
\]
as a power series. 


(b) What is the coefficient of $x^{3}$?

\SOLUTION



(a)
\begin{align*}
\left(
\frac{1}{2x - 3}
\right)^5
&= ??? \\
\end{align*}

ANSWER:
\boxed{???}
\qed


(b)

ANSWER:
\boxed{???}
\qed


\newpage




Q4.
(a) Rewrite
\[
\biggl( 
\frac{1}{1 - 4x^2}
\biggr)^5
\]
as a power series. 

(b) What is the coefficient of $x^{1000}$?

(c) What is the coefficient of $x^{1001}$?


\SOLUTION


(a)
\begin{align*}
\left(
\frac{1}{1 - 4x^2}
\right)^5
&= ??? \\
\end{align*}

ANSWER:
\boxed{???}
\qed


(b)

ANSWER:
\boxed{???}
\qed


\newpage




Q5.
(a) Solve for $A$ and $B$ where
\[
\frac{1}{(1 - 2x)(3x - 1)} = \frac{A}{1 - 2x} + \frac{B}{3x - 1}
\]

(b) Rewrite 
\[
f(x) = \frac{1}{(1 - 2x)(3x - 1)} 
\]
as a power series.

(c) What is the coefficient of $x^{1000}$ of $f(x)$?

\SOLUTION

(a) When we multiply
\[
\frac{1}{(1 - 2x)(3x - 1)} = \frac{A}{1 - 2x} + \frac{B}{3x - 1} \tag{1}
\]
with $(1 - 2x)(3x - 1)$, we get
\[
1 = ??? \tag{2}
\]

Substituting $x = ???$ into (2), we get
\begin{align*}
                 ??? &= ??? \\
\THEREFORE  \,\,\, A &= ??? \tag{3} 
\end{align*}
Substituting $x = ???$ into (2), we get
\begin{align*}
                  ??? &= ??? \\
\THEREFORE \,\,\, B   &= ??? \tag{4} 
\end{align*}

ANSWER:
\boxed{A=???, \hspace{1cm} B=???}
\qed


(b)
Substituting (3) and (4) into (1) we get
\begin{align*}
f(x) 
&= \frac{1}{(1 - 2x)(3x - 1)} 
\\
&= ??? \frac{1}{1 - 2x} + ??? \frac{1}{3x - 1} 
\\
&= ??? 
\\
&= ??? 
\\
&= ??? 
\\
\end{align*}

ANSWER:
\boxed{???}
\qed

(c)

ANSWER:
\boxed{???}
\qed





\end{document}
