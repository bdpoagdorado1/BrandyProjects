\newcommand\COURSENAME{Discrete Math 2}
\newcommand\COURSESHORTNAME{Discrete Math 2}
\newcommand\COURSENUMBER{MATH325}
\newcommand\TITLE{Assignment 10}
\newcommand\AUTHOR{Dr Yihsiang Liow}

\documentclass[a4paper,12pt]{article}
\usepackage{assignment}
\definecolor{red}{rgb}{1,0,0}
\newcommand\redtext[1]{\textcolor{red}{#1}}
\begin{document}
\topmatter

This assignment involves counting solutions
to linear systems using power series or generating
functions.


The goal is to study the number of solutions to the following system:
\[
\begin{cases}
a + b + c = n \\
a,b,c \in \Z \\
0 \leq a, 2 \leq b, 0 < c, c \text{ odd}
\end{cases}
\]
for different values of $n \geq 0$.
Let $a_n$ be the number of solutions to the above system.
For instance $a_{1000}$ is the number of solutions to the above when 
$n$ is $1000$.

To be more specific, $a_0$ is the number of solutions to
\[
\begin{cases}
a + b + c = 0 \\
a,b,c \in \Z \\
0 \leq a, 2 \leq b, 0 < c, c \text{ odd}
\end{cases}
\]
In this case, we can compute the value of $a_0$ quickly: $a_0 = 0$.
In fact, since $b \geq 2$ and $c \geq 1$, if $a + b + c  = n$ has a solution,
then $a + b + c \geq 3$.
We conclude immediately that
\[
a_0 = a_1 = a_2 = 0
\]
Clearly $a_3 = 1$ since the only solution to 
\[
\begin{cases}
a + b + c = 0 \\
a,b,c \in \Z \\
0 \leq a, 2 \leq b, 0 < c, c \text{ odd}
\end{cases}
\]
is
\[
a = 0, b = 2, c = 1
\]
It's easy to see that $a_4 = 2$:
\[
4 = 1 + 2 + 1 = 0 + 3 + 1
\]

Before going on, you should compute $a_5$ and $a_6$ to get a feel for the 
problem.
(You can either do this by hand or write a program to this.)
The values of $a_n$ for $n = 0, 1, 2, \ldots, 6$ is useful for checking
against the general result that you will be deriving.

Let $f(x) = \sum_{n=0}^\infty a_n x^n$.

You will see that we do not need to guess a formula for $a_n$
and then prove by mathematical induction.
Instead we will use the method of power series to compute a formula
for $a_n$.

\newpage
Q1. Write down $f(x)$, 
the product of 3 power series such that the coefficient of 
$x^n$ of this product is $a_n$. 
You need not simplify. 
For instance the following is a product of power series (which is not
the answer to this question!)
\[
(2 + 2x + 2x^2 + \cdots)
(0 + 1x + 2x^2 + \cdots)
(1 + 2x + 3x^2 + \cdots)
\]
You should write down explicitly at least 3 nonzero terms for each power 
series.

\SOLUTION


(a)
\begin{align*}
\frac{1}{1 - 2x}
&= ??? \\
\end{align*}

ANSWER:
\boxed{???}
\qed


(b)

ANSWER:
\boxed{???}
\qed



\newpage

Q2 Rewrite the product of power series in (a) as a rational function.
The denominator should be factorized
(for instance $(1 - x^2) = (1 - x)(1 + x)$), 
with similar terms collected together.
For instance the following is a rational function 
(which is not the right answer to this question!)
\[
f(x) = \frac{x}{(2 - 3x)^3(1 + 5x)^6}
\]
Note that all the $(1 + 5x)$ factors (there are 6 of them) 
are collected together.
Do {\it not} write for instance
\[
f(x) = \frac{x}{(2 - 3x)^3(1 + 5x)^2(1 + 5x)^4}
\]

\SOLUTION



\begin{align*}
R &= {\cal O}
\end{align*}





\newpage
Q3.
Using the theory of partial fractions,
rewrite the rational expression as a linear sum of simpler rational functions 
of the form $\frac{1}{p(x)^k}$ where $p(x)$ is a polynomial of degree
at most 2. 
For instance the following is a linear sum of such rational functions:
\[
f(x) = 5 \frac{1}{(1 - x)^3} + \frac{3}{4} \frac{1}{(2 + x + 3x^2)^5}
\]
(of course this is not the answer!)

[Hint: For this problem, the $p(x)$'s are in fact linear, i.e. degree $1$.
Also, there are 4 terms in this sum.]

\SOLUTION


(a)
\begin{align*}
\left(
\frac{1}{2x - 3}
\right)^5
&= ??? \\
\end{align*}

ANSWER:
\boxed{???}
\qed


(b)

ANSWER:
\boxed{???}
\qed






\newpage
Q4.
Using Q3, rewrite $f(x)$ as a power series with the coefficient of $x^n$
in terms of $n$.
The following is an example (which is of course not the right answer!)
\[
f(x) = \sum_{n=0}^\infty \frac{2 + n^2}{1 + n} x^n
\]
Simplify the coefficients of $x^n$ so that the binomial coefficients
(if any does) does not occur.
For instance you know that
\[
\binom{100}{3} = \frac{100 \cdot 99 \cdot 98}{1 \cdot 2 \cdot 3}
\]
and likewise
\[
\binom{n}{3} = \frac{(n-1) \cdot (n-2) \cdot (n-3)}{1 \cdot 2 \cdot 3}
\]
etc.
You are strongly advised to write some simple programs to check your
computation.
\SOLUTION


\begin{align*}
R &= {\cal O}
\end{align*}





\newpage
Q5. 
What is the coefficient $x^n$ from Q4 in terms of $n$
Recall that the coefficient of $x^n$ for the power series of
$f(x)$ is $a_n$.
In other words what is the formula for $a_n$ in terms of $n$?

[It's a good idea now to check the formula against the values of
$a_n$ for $n = 0, 1, 2, \ldots, 6$ that you have computed earlier.]

\SOLUTION

(a) When we multiply
\[
\frac{1}{(1 - 2x)(3x - 1)} = \frac{A}{1 - 2x} + \frac{B}{3x - 1} \tag{1}
\]
with $(1 - 2x)(3x - 1)$, we get
\[
1 = ??? \tag{2}
\]

Substituting $x = ???$ into (2), we get
\begin{align*}
                 ??? &= ??? \\
\THEREFORE  \,\,\, A &= ??? \tag{3} 
\end{align*}
Substituting $x = ???$ into (2), we get
\begin{align*}
                  ??? &= ??? \\
\THEREFORE \,\,\, B   &= ??? \tag{4} 
\end{align*}

ANSWER:
\boxed{A=???, \hspace{1cm} B=???}
\qed


(b)
Substituting (3) and (4) into (1) we get
\begin{align*}
f(x) 
&= \frac{1}{(1 - 2x)(3x - 1)} 
\\
&= ??? \frac{1}{1 - 2x} + ??? \frac{1}{3x - 1} 
\\
&= ??? 
\\
&= ??? 
\\
&= ??? 
\\
\end{align*}

ANSWER:
\boxed{???}
\qed

(c)

ANSWER:
\boxed{???}
\qed




\newpage
Q6.
(a) How many solutions are there to
\[
\begin{cases}
a + b + c = 1000 \\
a,b,c \in \Z \\
0 \leq a, 2 \leq b, 0 < c, c \text{ odd}
\end{cases}
\]

(b)
How many solutions are there to
\[
\begin{cases}
a + b + c = 1001 \\
a,b,c \in \Z \\
0 \leq a, 2 \leq b, 0 < c, c \text{ odd}
\end{cases}
\]


\SOLUTION

\input{q06.tex}




	
\end{document}
